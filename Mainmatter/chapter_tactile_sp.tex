\chapter{Predicción de Estabilidad de Agarres}

Cuando los humanos agarramos objetos, sabemos intuitivamente si el agarre es estable o no, incluso antes de levantar o manipular el objeto. No es necesario que levantemos las manos o apliquemos ninguna fuerza (ya sea directa, como golpear el objeto, o indirecta, como la gravedad) para determinar que el objeto permanecerá estable entre los dedos o las palmas de las manos. Aprovechando nuestro sentido táctil, junto con nuestra visión y toda nuestra intuición y años de experiencia, podemos predecir con precisión la estabilidad del agarre de una variedad de objetos en una amplia gama de situaciones. Esta habilidad es deseable para cualquier manipulador robótico ya que favorece la detección temprana de fallos de agarre para que el robot pueda reaccionar lo más rápido posible. Por ejemplo, un robot de reabastecimiento que trabaje en una tienda reconocería cuándo un objeto se puede deslizar de su mano y reaccionaría en consecuencia y rápidamente para evitar una situación en la que el objeto se caería y posiblemente se descompondría. Por muy fácil que sea esta tarea para un ser humano, no es tan obvia para un robot: simplemente hay demasiadas variables a tener en cuenta para una solución sencilla, por ejemplo, el peso, la rugosidad y la geometría del objeto; la fuerza de la gravedad; y el impulso que puede tener el objeto si lo movemos por nombrar algunas.

El problema de predecir la estabilidad de un agarre es un tema de investigación desafiante y continuo en el campo de la robótica. La gran mayoría de la literatura, que se revisará en profundidad más adelante en este capítulo, utiliza los sensores táctiles como la principal fuente de datos, ya que proporcionan información valiosa y abundante (por ejemplo, temperatura o presión) sobre las fuerzas que actúan durante la interacción de la mano robótica con los objetos \cite{Kappassov2015}. Aunque los métodos, conjuntos de datos y sensores varían según las obras existentes, todos ellos concuerdan en distinguir dos estados diferentes para el agarre: estable, lo que significa que el objeto está firmemente agarrado; o resbaladizo, lo que significa que el objeto podría deslizarse de la mano.

Trabajos anteriores encontrados en la literatura abordan este problema siguiendo la siguiente metodología: agarrar el objeto, leer los sensores táctiles equipados en los dedos y/o la palma de la mano, calcular características personalizadas que intentan caracterizar estos dos estados de estabilidad y aprenderlos para hacer predicciones futuras \cite{Li2014b,Dang2014,Su2015b,Veiga2015}. Estas propuestas tratan las lecturas táctiles como señales clásicas: las preprocesan como si fueran matrices, calculan características y aprenden sus características utilizando métodos probabilísticos. Como consecuencia, su rendimiento depende en gran medida de las características seleccionadas. Además, la distribución espacial inherente al sensor táctil se pierde debido al hecho de comprimir los datos en una matriz unidimensional.

En este capítulo, proponemos el uso de \acp{GNN} para predecir la estabilidad del agarre. Dado que se trata de modelos de aprendizaje profundo, no hay necesidad de utilizar funciones de ingeniería manual, ya que el algoritmo está diseñado para aprenderlos por sí mismo. Además, los gráficos pueden reflejar con mayor precisión la distribución real de los electrodos en el sensor, así como sus relaciones espaciales, que deberían ser de gran valor para el aprendizaje de las características táctiles. Las principales contribuciones de este capítulo pueden resumirse como sigue:

\begin{itemize}
	\item Procesamos las lecturas táctiles desde una perspectiva novedosa: en lugar de considerarlas como matrices 1D o imágenes 2D, construimos un gráfico 3D que conecta los múltiples puntos de detección (taxeles) del sensor táctil.

	\item Introducimos una forma novedosa de procesar dicha información utilizando \acfp{GNN}.

	\item Se comprueba cuantitativamente el rendimiento de esta nueva metodología en el mundo real mediante un conjunto de sensores táctiles instalados en una mano robótica.

	\item Lanzamos una extensión que efectivamente duplica el tamaño de un conjunto de datos ya existente \cite{Zapata2018} para la predicción de la estabilidad de agarre e incluye una nueva división completa para las pruebas.
\end{itemize}