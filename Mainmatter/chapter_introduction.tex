\chapter{Introduction}
\label{cha:introduction}

\begin{chapterabstract}
This first chapter introduces the main topic of this thesis. The chapter is organized into five different sections: Section \ref{cha:introduction:sec:motivation} establishes the framework for the research activity proposed in this thesis; Section \ref{cha:introduction:sec:approach} introduces the proposal developed during the thesis; Section \ref{cha:introduction:sec:contributions} states the contributions of this work; Section \ref{cha:introduction:sec:papers} provides a list of co-authored papers that were written during this thesis; finally, Section \ref{cha:introduction:sec:structure} details the structure of the document.
\end{chapterabstract}

\minitoc

\clearpage

\section{Introduction and Motivation}
\label{cha:introduction:sec:motivation}

Nowadays, one of the most important challenges of developed countries is stated in the Horizon 2020 guidelines as \emph{Health, Demographic Change, and Wellbeing}. The investment in this challenge aims to provide better health for everyone. It also focuses on keeping older people as active and independent as possible with novel and personalized health and care systems. By personalizing health and care, the aforementioned call commits to: "(1) improve our understanding of the causes and mechanisms underlying health, healthy aging, and disease; (2) improve our ability to monitor health and to prevent, detect, treat, and manage disease; (3) support older persons to remain active and healthy; and (4) test and demonstrate new models and tools for health and care delivery."

We focus on the last two points: supporting older persons and demonstrating new models and tools for care delivery. In that regard, this thesis has been conducted under the frame of two projects that have shaped its goals and directions to be aligned with the Horizon 2020 call: SIRMAVED (Development of an Integral Robotic System for Monitoring and Interaction with Acquired Brain Damage and Dependent People) and COMBAHO (Come Back Home System for Enhancing Autonomy of People with Acquired Brain Injury and Dependent to Integrate them Back into Society). Although the specifics for the projects are slightly different, both of them share a common goal: the development of a mobile robotic platform for delivering health care at home.

In order to accomplish the objectives of such projects, a cross-disciplinary effort is needed. Three fields must converge to provide an adequate solution: robotics, computer vision, and machine learning.

\begin{itemize}
	\item Robotics is a vast field which mainly deals with two different problems. First, the design, development, and construction of robot systems. Second, all the computer systems needed to control them and to provide them with sensorial feedback and the ability to process such information.

	\item If vision is the process of discovering and understanding the world through images, the goal of computer vision is to gift the machines with the ability to see, i.e., being able to produce high-level reasoning and understanding from digital images or videos. It deals with all the processes the human visual system would need to carry out: acquisition, processing, and analysis to understand such images.

	\item Tightly coupled with the two previous fields, machine learning is arguably the most difficult to define. It is often considered a branch or subset of the more general term Artificial Intelligence. In a broad sense, it aims to build mathematical models, which are tuned by experience, so that a computer may use them to carry out a certain task without having been explicitly programmed to do so.
\end{itemize}

By combining techniques and knowledge from those three areas, we might be able to achieve one of the milestones of the challenge posed by the robotic system: the need for full scene understanding. Despite the fact that there are other feats to complete the challenge, in this thesis we will place our attention on achieving a high-level understanding of the environment in which the robot must navigate and the objects with which it may interact. The general goal of this thesis is to advance and move towards full scene understading on a robotic platform so our approach is completely guided by this mindset.

\section{Approach}
\label{cha:introduction:sec:approach}

This work focuses on a subset of the problems that we stated in the previous section from a learning-based point of view:

\begin{itemize}
    \item Object Class Recognition.
    \item Semantic Segmentation.
    \item Simulation to Real Transfer.
    \item Tactile Sensing.
\end{itemize}

The main thread that stitches all the components of this thesis together is deep learning. While conventional machine learning methods were limited to process data at its rawest form, i.e., they required crafting features manually for the learning algorithm to process them, deep learning allows such models to learn representations for the data automatically without any need for hand-engineering or domain expertise. At its heart, a deep learning model is a representation learning method with various representation levels which are generated by composing linear and non-linear layers that transform the representations into higher and generally more abstract ones in a hierarchical fashion \cite{Lecun2015}. Such approach has surpassed traditional methods in a vast array of fields such as image processing, speech, and natural language processing.

\newpage

Apart from taking advantage of deep learning as a tool to solve those problems, this thesis places emphasis on the data themselves and novel ways to process them:

\begin{itemize}
    \item We make use of \ac{3D} information and make it the core element of all our approaches under the assumption that the additional dimension can be useful to learn to classify and segment without the ambiguity of \ac{2D} representations.
    \item We introduce learning-based architectures that are able to process such \ac{3D} information respecting its spatial arrangement. Other works have made use of \ac{3D} data but using \ac{2D} representations for it thus not taking advantage of the rich spatial information in the extra dimension.
\end{itemize}

\section{Contributions}
\label{cha:introduction:sec:contributions}

As we already stated, this work concentrates on pushing forward three key aspects of robotic perception: object classification, semantic segmentation, and tactile sensing. In this regard, the contributions of this thesis stem from such areas and are as follows:

\begin{itemize}
    \item We propose a \acl{CNN} architecture for \acs{3D} object classification which makes use of \acs{3D} representations such as point clouds or meshes by structuring them into a voxel grid. Futhermore, it is tested under difficult conditions such as noise and occlusion to gain insight about real-world situations. We also iterate over that initial architecture, creating a novel slice-based model which significantly improves over other approaches. We show the performance of these models and prove their suitability for real time object classification.
    \item We perform a comprehensive review of the state of the art of semantic segmentation for image and video using deep learning techniques. In such review, apart from providing details about all existing methods and datasets, we also gather insight about weaknesses and future research.
    \item Following that train of thought, we introduce a novel large-scale dataset for various robotic perception tasks with special focus on 3D semantic segmentation.
    \item Finally, we show a novel \acl{GNN} architecture for tactile sensing which is able to classify the stability of robotic grasps using humanoid hands equipped with tactile sensors whose readings are interpreted as \acs{3D} graphs.
\end{itemize}

\section{Co-Authored Papers}
\label{cha:introduction:sec:papers}

This thesis is the result of continuous effort throughout the last years. Such efforts have sometimes crystallized in form of journal publications, conference talks, and poster presentations. A significant part of this thesis consists of extracts from the following co-authored publications.

\subsection{Chapter \ref{cha:objrecog}: 3D Object Classification}

\begin{itemize}
  \item \fullcite{Garcia-Garcia2016c}
  \item \fullcite{Garcia-Garcia2017}
  \item \fullcite{Gomez-Donoso2017b}
\end{itemize}

\subsection{Chapter \ref{cha:semseg}: Semantic Segmentation}

\begin{itemize}
  \item \fullcite{Garcia-Garcia2017b}
\end{itemize}

\subsection{Chapter \ref{cha:sim2real}: Sim2Real}

\begin{itemize}
    \item \fullcite{Garcia-Garcia2018}
    \item \fullcite{Martinez-Gonzalez2018}
    \item \fullcite{Oprea2019}
    \item \fullcite{Garcia-Garcia2019b}
  \end{itemize}

\subsection{Chapter \ref{cha:tactile}: Tactile Sensing}

\begin{itemize}
  \item \fullcite{Garcia-Garcia2019}
  \item \fullcite{Stiven-Zapata2019}
\end{itemize}

\subsection{Other}

During the years spent working on the main topics of this thesis, several collaborations and side works were carried out that also were published either as journal papers, conference proceedings, or preprints. Those works, although not strictly related to the content of this thesis, helped in various ways: exchanging ideas that later inspired other concepts, sparking collaborations, and also expanding the knowledge of other interesting areas of research.

\begin{itemize}
  \item \fullcite{Oprea2017b}
  \item \fullcite{Gomez-Donoso2017}
  \item \fullcite{Oprea2017}
  \item \fullcite{Garcia-Garcia2016d}
  \item \fullcite{Garcia-Garcia2016b}
  \item \fullcite{Saval-Calvo2016}
  \item \fullcite{Orts-Escolano2016}
  \item \fullcite{Garcia-Garcia2016}
  \item \fullcite{Mora2016}
  \item \fullcite{Orts-Escolano2015}
  \item \fullcite{Orts-Escolano2014}
\end{itemize}

\section{Thesis Structure}
\label{cha:introduction:sec:structure}

This thesis is structured as follows. This first chapter introduced the problem and the motivation for this work; it also stated the goals and contributions. The following four chapters discuss the four core problems of this thesis: 3D object classification in Chapter \ref{cha:objrecog}, semantic segmentation in Chapter \ref{cha:semseg}, simulation to real transfer in Chapter \ref{cha:sim2real}, and tactile sensing in Chapter \ref{cha:tactile}. For each chapter, we review the state of the art, we describe our proposal, and we also carry out experiments to validate such contributions. Finally, in Chapter \ref{cha:conclusion}, we draw overall conclusions, revisit the highlights of this work, and discuss the possible application scenarios as well as future research directions.