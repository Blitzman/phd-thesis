\chapter{Introduction}
\label{cha:introduction}

\begin{chapterabstract}
\end{chapterabstract}

\minitoc

\clearpage

\section{Motivation}
\label{cha:introduction:sec:motivation}

\section{Approach}
\label{cha:introduction:sec:approach}

The main thread that stitches all the components of this thesis together is deep learning.

This work focuses on a subset of the problems that we stated in the previous section from a learning-based point of view:

\begin{itemize}
    \item Object Class Recognition.
    \item Semantic Segmentation.
    \item Simulation to Real Transfer.
    \item Tactile Sensing.
\end{itemize}

Apart from taking advantage of deep learning as a tool to solve those problems, this thesis places emphasis on the data themselves and novel ways to process them:

\begin{itemize}
    \item We make use of \ac{3D} information and make it the core element of all our approaches. The observation is that the additional dimension can be useful to learn to classify and segment without the ambiguity of \ac{2D} representations.
    \item We introduce learning-based architectures that are able to process such \ac{3D} information respecting its spatial arrangement. If we were to process \ac{3D} data in a \ac{2D} way we would start giving away all its advantages.
\end{itemize}

\subsection{Machine Learning in Computer Vision}

\subsection{Machine Learning in Robotics}

\section{Contributions}
\label{cha:introduction:sec:contributions}

As we already stated, this work concentrates on pushing forward three key aspects of robotic perception: object classification, semantic segmentation, and tactile sensing. In this regard, the contributions of this thesis stem from such areas and are as follows:

\begin{itemize}
    \item We propose a \acl{CNN} architecture for \acs{3D} object classification which makes use of \acs{3D} representations such as point clouds or meshes by structuring them into a voxel grid. Futhermore, it is tested under difficult conditions such as noise and occlusion to gain insight about real-world situations. We also iterate over that initial architecture, creating a novel slice-based model which significantly improves over other approaches. We show the performance of these models and prove their suitability for real time object classification.
    \item We perform a comprehensive review of the state of the art of semantic segmentation for image and video using deep learning techniques. In such review, apart from providing details about all existing methods and datasets, we also gather insight about weaknesses and future research.
    \item Following that train of thought, we introduce a novel large-scale dataset for various robotic perception tasks with special focus on 3D semantic segmentation.
    \item Finally, we show a novel \acl{GNN} architecture for tactile sensing which is able to classify the stability of robotic grasps using humanoid hands equipped with tactile sensors whose readings are interpreted as \acs{3D} graphs.
\end{itemize}

\section{Co-Authored Papers}
\label{cha:introduction:sec:papers}

This thesis is the result of continuous effort throughout the last years. Such efforts have sometimes crystallized in form of journal publications, conference talks, and poster presentations. A significant part of this thesis consists of extracts from the following co-authored publications.

\subsection{Chapter \ref{cha:objrecog}: 3D Object Classification}

\begin{itemize}
  \item \fullcite{Garcia-Garcia2016c}
  \item \fullcite{Garcia-Garcia2017}
  \item \fullcite{Gomez-Donoso2017b}
\end{itemize}

\subsection{Chapter \ref{cha:semseg}: Semantic Segmentation}

\begin{itemize}
  \item \fullcite{Garcia-Garcia2017b}
\end{itemize}

\subsection{Chapter \ref{cha:sim2real}: Sim2Real}

\begin{itemize}
    \item \fullcite{Garcia-Garcia2018}
    \item \fullcite{Martinez-Gonzalez2018}
    \item \fullcite{Oprea2019}
    \item \fullcite{Garcia-Garcia2019b}
  \end{itemize}

\subsection{Chapter \ref{cha:tactile}: Tactile Sensing}

\begin{itemize}
  \item \fullcite{Garcia-Garcia2019}
  \item \fullcite{Stiven-Zapata2019}
\end{itemize}

\subsection{Other}

During the years spent working on the main topics of this thesis, several collaborations and side works were carried out that also were published either as journal papers, conference proceedings, or preprints. Those works, although not strictly related to the content of this thesis, helped in various ways: exchanging ideas that later inspired other concepts, sparking collaborations, and also expanding the knowledge of other interesting areas of research.

\begin{itemize}
  \item \fullcite{Oprea2017b}
  \item \fullcite{Gomez-Donoso2017}
  \item \fullcite{Oprea2017}
  \item \fullcite{Garcia-Garcia2016d}
  \item \fullcite{Garcia-Garcia2016b}
  \item \fullcite{Saval-Calvo2016}
  \item \fullcite{Orts-Escolano2016}
  \item \fullcite{Garcia-Garcia2016}
  \item \fullcite{Mora2016}
  \item \fullcite{Orts-Escolano2015}
  \item \fullcite{Orts-Escolano2014}
\end{itemize}

\section{Thesis Structure}
\label{cha:introduction:sec:structure}

This thesis is structured as follows. This first chapter introduced the problem and the motivation for this work; it also stated the goals and contributions. The following four chapters discuss the four core problems of this thesis: 3D object classification in Chapter \ref{cha:objrecog}, semantic segmentation in Chapter \ref{cha:semseg}, simulation to real transfer in Chapter \ref{cha:sim2real}, and tactile sensing in Chapter \ref{cha:tactile}. For each chapter, we review the state of the art, we describe our proposal, and we also carry out experiments to validate such contributions. Finally, in Chapter \ref{cha:conclusion}, we draw overall conclusions, revisit the highlights of this work, and discuss the possible application scenarios as well as future research directions.