\chapter{Clasificación de Objetos}

La clasificación de objetos es fundamental para la visión por computador y, a pesar de los avances logrados en los últimos años, sigue siendo un área de investigación difícil. Podría decirse que la mayor parte del interés en la clasificación de objetos se debe a su utilidad para la robótica en situaciones del mundo real y cotidiano.

En este sentido, el reconocimiento de los objetos es uno de los problemas que hay que resolver para lograr una comprensión total de la escena visual. Este conocimiento más profundo y mejor del entorno facilita y permite la ejecución de una amplia variedad de tareas más complejas. Por ejemplo, el reconocimiento preciso de objetos en una habitación puede ser extremadamente útil para cualquier sistema robótico que navega en ambientes interiores para ubicarse a sí mismo o para realizar tareas dentro de tales escenarios. Debido a la naturaleza no estructurada de esos entornos, los robots autónomos necesitan hacer razonamientos basados en el mundo real dinámico. En otras palabras, necesitan entender la información capturada por sus sensores para realizar tareas como agarrar, navegar, cartografiar, o incluso proporcionar a los seres humanos información sobre su entorno. Identificar las clases de los objetos es un paso clave para mejorar las capacidades mencionadas anteriormente.

A pesar de la fácil e intuitiva interpretación del problema, su dificultad inherente puede ser engañosa. Nosotros, los humanos, reconocemos numerosos objetos en entornos difíciles (por ejemplo, diferentes puntos de vista, oclusión o desorden) con poco o ningún esfuerzo. Sin embargo, abordar este problema no es tan fácil para un ordenador y teniendo en cuenta todos los posibles ajustes y combinaciones de factores externos hace que esta tarea sea difícil de resolver de forma eficiente y con gran precisión (lo que a menudo se requiere en numerosos escenarios de aplicación).

La mayor parte de la literatura clásica sobre este tema abordó este problema ideando descriptores de características hechos a mano que se extraen en ciertos puntos clave detectados sobre la imagen bidimensional y que luego se utilizan para compararlos con descriptores de objetos preexistentes en una base de datos para hacerlos coincidir con una determinada clase, o bien para alimentarlos como entrada a una arquitectura de aprendizaje de máquina poco profunda que aprende a clasificar esos descriptores para predecir la clase del objeto que aparece en la imagen. Ese paradigma cambió recientemente debido al éxito de las arquitecturas de aprendizaje profundo que son capaces de explotar sus capacidades de aprendizaje de características para evitar la necesidad de descriptores manuales de ingeniería mientras logran niveles de precisión sin precedentes. Además, la adopción y difusión de los sensores de profundidad también ha añadido una nueva dimensión de la que aprender para mejorar el rendimiento. Los enfoques introducidos en esta tesis forman parte de esa tendencia de vanguardia que aprovecha la información geométrica adicional facilitada por los escáneres de rango de productos básicos para realizar el aprendizaje sobre ellos utilizando arquitecturas profundas.

Aparte de los métodos, los datos también desempeñan un papel clave en la clasificación de objetos. A medida que los métodos han evolucionado, también lo han hecho los conjuntos de datos. Debido a las crecientes necesidades impuestas por los enfoques basados en datos, los conjuntos de datos se han hecho más grandes, variados y ricos. Ese progreso ha permitido el desarrollo de nuevas formas de resolver el problema del reconocimiento de clases de objetos, por ejemplo, utilizando datos tridimensionales.

El primer enfoque que proponemos para realizar la clasificación de objetos utilizando datos 3D, PointNet, es capaz de aprender clases de objetos a partir de nubes de puntos discretizadas como cuadrículas de ocupación con cuadrículas de vóxeles uniformes en el espacio tridimensional. A continuación, analizamos cómo el ruido y la oclusión impactan esta arquitectura de aprendizaje profundo en 3D y la importancia de la representación de los datos cuando se trata de condiciones tan adversas que comúnmente aparecen en el mundo real. En ese estudio, también proponemos cambios menores a la arquitectura y a la representación en sí que aumentan significativamente la precisión con respecto a la propuesta original de PointNet. Por último, presentamos una novedosa arquitectura basada en cortes para abordar el problema del reconocimiento de clases de objetos 3D, LonchaNet, que ha conseguido resultados de última generación en un punto de referencia estándar.