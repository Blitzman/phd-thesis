\chapter{Segmentación Semántica}

Hoy en día, la segmentación semántica -aplicada a imágenes 2D, vídeo e incluso datos volumétricos o 3D - es uno de los problemas clave en el campo de la visión por computador. A grandes rasgos, la segmentación semántica es una de las tareas de alto nivel que allana el camino hacia la comprensión completa de la escena. La importancia de la comprensión de las escenas como problema central de la visión por computador se pone de manifiesto por el hecho de que un número cada vez mayor de aplicaciones se nutre de la inferencia de conocimiento a partir de imágenes. Algunas de dichas aplicaciones incluyen la conducción autónoma, la interacción hombre-máquina, la fotografía computacional, los motores de búsqueda de imágenes y la realidad aumentada, por nombrar algunos. Este problema se ha abordado en el pasado utilizando diversas técnicas tradicionales de visión por computador y el aprendizaje automático. A pesar de la popularidad de este tipo de métodos, la revolución del aprendizaje profundo ha cambiado el \emph{status quo} de tal manera que muchos de los problemas de la visión por computador -- la segmentación semántica entre ellos -- están siendo abordados usando arquitecturas profundas, usualmente \acp{CNN}. \cite{Ning2005, Ciresan2012, Farabet2013, Hariharan2014, Gupta2014}, que están superando a otros enfoques por un gran margen en términos de precisión y a veces incluso de eficiencia. Sin embargo, el aprendizaje profundo está lejos de la madurez alcanzada por otras ramas más tradicionales de la visión por computador y el aprendizaje automático. Debido a esto, existe una carencia de obras unificadoras y de revisiones del estado del arte. El estado siempre cambiante del campo dificulta la iniciación y mantener el ritmo de su evolución es una tarea que consume mucho tiempo debido a la gran cantidad de literatura nueva que se genera. Esto dificulta el seguimiento de los trabajos relacionados con la segmentación semántica y la correcta interpretación de sus propuestas y la validación de los resultados.

Existen varios estudios de segmentación semántica como los trabajos de Zhu \emph{et al.}\cite{Zhu2016} y Thoma\cite{Thoma2016}, que realizan un gran trabajo resumiendo y clasificando los métodos existentes, discutiendo conjuntos de datos y métricas y proporcionando opciones para futuras líneas de investigación. Sin embargo, carecen de algunos de los conjuntos de datos más recientes, no analizan todos los métodos y ninguno de ellos proporciona detalles minuciosos sobre las técnicas de aprendizaje profundo. Las principales contribuciones de nuestro trabajo son las siguientes:

\begin{itemize}
	\item Proporcionamos un amplio estudio de los conjuntos de datos existentes que pueden ser útiles para proyectos de segmentación con técnicas de aprendizaje profundo.
	\item Una revisión en profundidad y organizada de los métodos más significativos que utilizan el aprendizaje profundo para la segmentación semántica, sus orígenes y sus contribuciones.
	\item Una evaluación completa del rendimiento que recoge métricas cuantitativas como la precisión y el tiempo de ejecución.
	\item Discusión sobre los resultados antes mencionados, así como una lista de posibles trabajos futuros que podrían marcar el curso de los próximos avances y una conclusión que resume el estado del arte en este campo.
\end{itemize}
