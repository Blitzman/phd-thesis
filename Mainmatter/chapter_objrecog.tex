\chapter{Object Recognition}
\label{cha:objrecog}

\begin{chapterabstract}
In this chapter, we address the problem of object class recognition. To approach this challenge, we rely on the geometric information provided by 3D object representations such as point clouds. Furthermore, we focus on learning-based methods to distinguish objects from different classes while capturing the variability of shape of different objects which belong to the same class. More specifically, we leverage deep learning for such task. The chapter begins introducing and formulating the object recognition task in Section \ref{cha:objrecog:sec:introduction} followed by a review of the most relevant literature in Section \ref{cha:objrecog:sec:relatedworks}. After that, we present our first proposal towards 3D object recognition using \acp{CNN}, namely PointNet, in Section \ref{cha:objrecog:sec:pointnet}. Later, PointNet is improved and thoroughly tested in adverse conditions with noise and occlusion throughout the study in Section \ref{cha:objrecog:sec:study}. Next, LonchaNet is introduced in Section \ref{cha:objrecog:sec:lonchanet} as the last iteration of our system that incorporates all the lessons learned by the previous work. Finally, Section \ref{cha:objrecog:sec:conclusion} draws conclusions and sets future lines of research.
\end{chapterabstract}

\section{Introduction}
\label{cha:objrecog:sec:introduction}

Object recognition is fundamental to computer vision and despite the progress achieved during the last years, it still remains a challenging area of research. Arguably, most of the interest in object recognition is due to its usefulness for robotics.

In that regard, recognizing objects is one of the problems that must be solved to achieve total visual scene understanding. Such deeper and better knowledge of the environment eases and enables the execution of a wide variety of more complex tasks. For instance, accurately recognizing objects in a room can be extremely useful for any robotic system that navigates within indoor environments. Due to the unstructured nature of those environments, autonomous robots need to do reasoning grounded in the dynamic real world. In other words, they need to understand the information captured by their sensors to perform tasks such as grasping, navigation, mapping, or even providing humans with information about their surroundings. Identifying the classes to which objects belong is one key step to enhance the aforementioned capabilities.

Despite the easy intuitive interpretation of the problem, its inherent difficulty can be misleading. We humans recognize numerous objects in difficult settings (e.g., different points of view, occlusion, or clutter) with little to no effort. However, approaching that problem is not that easy for a computer and taking into account all the possible settings and combinations of external factors renders this task a difficult one to solve efficiently and with high precision (which is often required in numerous application scenarios).

From a formal point of view, the object recognition task can be formulated as follows: given an image $\mathcal{I}^{H\times W}$ in which an object $\mathcal{O}$ appears, which can be either a gray-scale or RGB array of $W$ pixels in width and $H$ pixels in height, the goal is to predict the class of the object $\mathcal{L_O}$ from a set of $N$ predefined object classes $\mathcal{L} = \{\mathcal{L}_0, \mathcal{L}_1, ..., \mathcal{L}_{N-1}\}$.

For those reasons, object recognition lies at the intersection of computer vision, robotics, and machine learning. That fact makes it one of the most active and important fields at the moment.

\section{Related Works}
\label{cha:objrecog:sec:relatedworks}

\subsection{2D Object Recognition}
\label{cha:objrecog:sec:relatedworks:subsec:2d}

\subsection{RGB-D Object Recognition}
\label{cha:objrecog:sec:relatedworks:subsec:rgbd}

\subsection{3D Object Recognition}
\label{cha:objrecog:sec:relatedworks:subsec:3d}

\section{PointNet}
\label{cha:objrecog:sec:pointnet}

\subsection{Data Representation}
\label{cha:objrecog:sec:pointnet:subsec:data}

The system takes a point cloud of an object as input to recognize it, i.e., predict its class label. However, point clouds are unstructured representations that cannot be easily handled by common \ac{CNN} architectures due to the lack of a matrix-like organization.

\subsection{Network Architecture}
\label{cha:objrecog:sec:pointnet:subsec:network}

\subsection{Experiments}
\label{cha:objrecog:sec:pointnet:subsec:experiments}

\subsection{Discussion}
\label{cha:objrecog:sec:pointnet:subsec:discussion}

\section{Noise and Occlusion}
\label{cha:objrecog:sec:study}

\section{LonchaNet}
\label{cha:objrecog:sec:lonchanet}

\section{Conclusion}
\label{cha:objrecog:sec:conclusion}
