\chapter{Resumen y Conclusión}
\label{cha:conclusion_sp}

La atención a las personas dependientes (por razones de envejecimiento, accidentes, discapacidades o enfermedades) es una de las líneas de investigación prioritarias para los países europeos, tal y como se recoge en los objetivos del marco Horizonte 2020. Con el fin de minimizar el coste y la intrusividad de las terapias para el cuidado y la rehabilitación, se desea que tales cuidados sean administrados en el hogar del paciente. La solución natural para este entorno es una plataforma robótica móvil en interiores.

Esta plataforma robótica para el cuidado en el hogar necesita resolver hasta cierto punto un conjunto de problemas que se encuentran en la intersección de múltiples disciplinas, por ejemplo, la visión por computador, el aprendizaje por computadora y la robótica. En esa encrucijada, uno de los retos más notables (y en el que nos centraremos) es la comprensión de la escena: el robot necesita entender el entorno desestructurado y dinámico en el que navega y los objetos con los que puede interactuar.

Para lograr una comprensión completa de la escena, se deben realizar varias tareas. En esta tesis nos centraremos en tres de ellas: reconocimiento de la clase de objetos o \emph{object recognition}, segmentación semántica o \emph{semantic segmentation} y predicción de la estabilidad del agarra de los objetos (también denominado \emph{grasp stability prediction}). La primera se refiere al proceso de categorizar un objeto de acuerdo a un conjunto de clases (por ejemplo, silla, cama o almohada); la segunda va un nivel más allá de la categorización de objetos y tiene como objetivo proporcionar un etiquetado denso por píxel de cada objeto en una imagen; la tercera consiste en determinar si un objeto que ha sido agarrado por una mano robótica está en una configuración estable o si va a caer o deslizarse.

Esta tesis presenta contribuciones a la resolución de estas tres tareas utilizando el aprendizaje profundo (\emph{deep learning}) como la metodología principal para resolver estos problemas de reconocimiento, segmentación y predicción. Todas estas soluciones comparten una observación central: todas se basan en datos tridimensionales para aprovechar esa dimensión adicional y su disposición espacial. Las cuatro contribuciones principales de esta tesis son: en primer lugar, mostramos un conjunto de arquitecturas y representaciones de datos para la clasificación de objetos 3D utilizando nubes de puntos; en segundo lugar, llevamos a cabo una extensa revisión del estado del arte de los conjuntos de datos y métodos de segmentación semántica; en tercer lugar, introducimos un novedoso conjunto de datos sintéticos y a gran escala para la resolución conjunta de diversos problemas de robótica y visión; por último, proponemos un método y una representación alternativa para tratar con los sensores táctiles y aprender a predecir la estabilidad de agarre.

\section{Conclusiones}
\label{cha:conclusion_sp:sec:findings}

En esta tesis, hemos explorado cuatro problemas centrales de la robótica en interiores con un fuerte énfasis en la visión por computador. Sin embargo, a pesar de las diferencias obvias y de la distancia entre cada tema, podemos extraer ciertas conclusiones generales.

En primer lugar, después de revisar el estado del arte de cada uno de los temas de esta tesis, podemos afirmar que el aprendizaje profundo ha cosechado suficientes éxitos con facilidad para ser considerado el paradigma por defecto para la resolución de problemas de visión por computador y robótica. En casi todos los aspectos, las arquitecturas de aprendizaje profundo superan a los enfoques tradicionales.

Una observación general para todos los problemas es el hecho de que las representaciones \ac{3D} ofrecen información útil que es claramente beneficiosa para modelos de aprendizaje (generalmente en forma de mejor capacidad de generalización). Sin embargo, para aprovechar plenamente la información tridimensional, las arquitectura deben tener debidamente en cuenta la disposición espacial y el carácter no estructurado de los datos \ac{3D}.

Siguiendo con la última observación, también experimentamos las dificultades del entrenamiento de modelos para los datos tridimensionales. Esta dimensión adicional plantea varios retos que van desde el consumo de memoria hasta el aumento del tiempo de ejecución, pasando por un entrenamiento más difícil y propenso al \emph{overfitting}. En ese sentido, para garantizar la generalización, se necesitan más datos y más variados.

Es exactamente en este punto donde la generación de datos sintéticos ha demostrado ser una herramienta extremadamente valiosa para evitar el coste y los errores de los conjuntos de datos etiquetados manualmente. Sin embargo, todavía tienen que lidiar con sus propios problemas, principalmente la transferencia del conocimiento adquirido en el dominio sintético al mundo real.

\section{Contribuciones}
\label{cha:conclusion_sp:sec:contributions}

Una vez que hemos establecido las conclusiones generales de la tesis, resumimos brevemente las principales contribuciones y hallazgos.

En el capítulo \ref{cha:objrecog} se presenta un conjunto de redes y representaciones de datos en 3D para el problema de la clasificación de objetos. Primero, propusimos \emph{PointNet}: una arquitectura para la clasificación de objetos \acs{3D} que hace uso de representaciones \acs{3D} como nubes de puntos o mallas al estructurarlas en una malla de ocupación. Más tarde, analizamos una versión mejorada bajo condiciones difíciles como el ruido y la oclusión para obtener información sobre situaciones del mundo real. Por último, presentamos nuestra última iteración, \emph{LonchaNet}, un nuevo modelo basado en cortes o \emph{slices} que mejora significativamente con respecto a otros enfoques. Mostramos el rendimiento de estos modelos y demostramos su idoneidad para la clasificación de objetos en tiempo real.

En el capítulo \ref{cha:semseg} realizamos una revisión exhaustiva del estado del arte de la segmentación semántica para imagen y vídeo utilizando técnicas de aprendizaje profundo. Proporcionamos un punto de partida útil para los investigadores noveles y también suficientes detalles y profundidad para que los más experimentados pudieran encontrarlo útil. Lo más importante es que, además de revisar los conjuntos de datos y los métodos, también recogemos información sobre las debilidades y fortalezas de los mismos así como líneas futuras de investigación.

Una de las observaciones clave después de la revisión fue la escasez de conjuntos de datos de alta calidad y de gran escala para el aprendizaje en algoritmos basados en datos \ac{3D}. Siguiendo ese hilo de pensamiento, el capítulo \ref{cha:sim2real} introdujo un nuevo conjunto de datos sintético de gran escala para diversas tareas de percepción robótica con especial atención a la segmentación semántica 3D: \emph{the RobotriX}. Junto con los datos, también se hizo público el conjunto completo de herramientas para generarlos con documentación detallada.

Finalmente, el capítulo \ref{cha:tactile} abordó el campo de \emph{tactile sensing} y mostró una arquitectura capaz de clasificar la estabilidad de los agarres robóticos utilizando manos humanoides equipadas con sensores táctiles cuyas lecturas se interpretan como grafos \acs{3D}. También demostramos que este enfoque exhibe mejor capacidad de generalización que los métodos anteriores basados en arquitecturas tradicionales.

\section{Limitaciones y Trabajos Futuros}
\label{cha:conclusion_sp:sec:future}

Para concluir esta tesis, hay muchos aspectos de los trabajos aquí presentados que pueden ser mejorados de una manera u otra. Aparte de las limitaciones que hay que abordar, la labor realizada aquí también plantea cuestiones que podrían suscitar importantes trabajos futuros. Aquí destacamos brevemente tanto las limitaciones como dichas líneas futuras de investigación.

\subsection{Clasificación de Objetos}

\begin{itemize}
    \item Representaciones eficientes y precisas de datos 3D: aunque las mallas de ocupación han demostrado ser útiles para la tarea de clasificación de objetos, todavía presentan una serie de desventajas que las hacen inadecuadas para ciertas situaciones. Una de las principales debilidades de dicha representación es el consumo de memoria (o complejidad espacial) que provoca modelos más grandes que apenas caben en la memoria de la \ac{GPU} para su entrenamiento si se requiere una resolución considerable. Representaciones alternativas dispersas, como \emph{octrees} o grafos, podrían ayudar, pero requerirían nuevas arquitecturas para su procesamiento.
    \item Conjuntos de datos del mundo real: nuestro trabajo se ha centrado principalmente en el desarrollo de algoritmos de reconocimiento de objetos bajo ciertas condiciones (objetos totalmente aislados y modelos sintéticos). Aunque experimentamos con la introducción de condiciones reales como el ruido y las oclusiones, todavía es cuestionable si los modelos propuestos se comportarán correctamente en el mundo real. Sería interesante comprobar si tales modelos pueden ser entrenados o al menos afinados con bases de datos de objetos del mundo real de menor escala para medir su capacidad de generalización.
\end{itemize}

\subsection{Segmentación Semántica}

\begin{itemize}
    \item Actualización de datasets y métodos: el aprendizaje profundo evoluciona a la velocidad de la luz y al ser un tema tan candente se dificulta el mantener una revisión del estado del arte actualizada. Desde la escritura de esta tesis se han publicado numerosos trabajos: conjuntos de datos y entornos novedosos a la vez que varios métodos, bien sea iteraciones de algunos ya existentes o conceptos radicalmente diferentes.
    \item Segmentación de panoramas e imágenes hiperespectrales: a pesar de la exhaustividad de la revisión del estado del arte, ciertos trabajos fueron apartados intencionalmente debido a la escasa relevancia que presentaban en dicho momento. Algunas de dichas líneas de investigación incluyen la segmentación de imágenes panorámicas e hiperespectrales, la particularidad es que ambas hacen uso de entradas diferentes a las imágenes RGB comunes.
    \item Segmentación en tiempo real: con la creciente complejidad de los modelos basados en aprendizaje, cada vez se están convirtiendo en más precisos pero al mismo tiempo más pesados y lentos. La mayoría de los trabajos se han centrado en incrementar la precisión sin tener en cuenta que dichos modelos puedan ser desplegados en dispositivos con escasa capacidad de cómputo para ser útiles en situaciones reales. Trabajos recientes tratan de seguir esta línea para optimizar modelos de segmentación con el objetivo de poderlos ejecutar en tiempo real pero manteniendo a su vez unas cotas de precisión adecuadas.
\end{itemize}

\subsection{Simulación a Real (Sim2Real)}

\begin{itemize}
    \item Deformaciones no rígidas: una limitación de nuestro conjunto de datos y de la propia herramienta desarrollada para generarlo es que las interacciones están restringidas a objetos rígidos. Los objetos no rígidos no son modelados, por lo que no se pueden producir deformaciones. Este es un problema difícil de abordar que podría aumentar la utilidad del conjunto de datos y de la herramienta de generación para muchas otras tareas de manipulación robótica.
    \item Generación de escenas aleatorias: uno de los principales defectos de nuestro conjunto de datos es la cantidad limitada de escenas fotorrealistas. Esto se debe a la dificultad de diseñar una escena plausible y al mismo tiempo asegurarse de que la iluminación, las texturas y la geometría sean lo más realistas posible. Una posible forma de mejorar o de establecer un conjunto de datos sería idear un enfoque algorítmico para generar este tipo de escenas.
    \item Post-procesamiento para un mayor realismo: recientemente, muchos trabajos han aprovechado las \acp{GAN} para aumentar los conjuntos de datos con cambios sutiles pero realistas. Nos preguntamos si este tipo de técnicas podrían utilizarse para mejorar el realismo de las imágenes renderizadas mediante \emph{adversarial training}.
    \item \emph{Ray-tracing} en tiempo real: con la aparición de generaciones modernas de \acp{GPU}, que cuentan con hardware diseñado explícitamente para realizar operaciones de trazado de rayos, sería posible utilizar esta tecnología en la fase de generación de datos para hacer que las escenas parezcan aún más realistas. Además de la integración con el motor en cuestión, también implica trabajo en la parte artística (texturizado y configuración de la iluminación).
\end{itemize}

\subsection{Tactile Sensing}

\begin{itemize}
    \item Topologías de grafos para las lecturas de los sensores: aunque se ha demostrado que es útil generar una representación en forma de grafo que represente mejor la topología real de los \emph{tactels} en un sensor táctil, queda una cuestión abierta: la existencia o no de representaciones más adecuadas. Se pueden llevar a cabo más experimentos para determinar esto. Una línea posible es aprender la topología misma a través de una red neuronal, por ejemplo, mediante una \ac{GNG} \cite{Fritzke1999}.
    \item Información temporal para agarres dinámicos: nuestra propuesta se centraba en detectar si un agarre estático sería estable o no. Sin embargo, los objetos rara vez son entidades estáticas sino que se ven afectados por muchas fuerzas diferentes. A ese respecto, una prometedora línea de investigación futura consiste en estudiar la posibilidad de añadir información temporal a las arquitecturas basadas en grafos. Una posible forma de hacerlo sería integrar otras arquitecturas recurrentes comunes para tal fin, por ejemplo \aclp{LSTM}.
\end{itemize}